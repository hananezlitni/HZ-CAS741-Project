\documentclass[12pt, titlepage]{article}

\usepackage{xr}
\usepackage{placeins}
\usepackage{booktabs}
\usepackage{tabularx}
\usepackage{hyperref}
\hypersetup{
    colorlinks,
    citecolor=blue,
    filecolor=magenta,
    linkcolor=black,
    urlcolor=cyan
}
\usepackage[round]{natbib}

\newcounter{testnum} %Test Number
\newcommand{\dthetestnum}{T\thetestnum}
\newcommand{\tref}[1]{T\ref{#1}}

\input{../../Comments}

\externaldocument{../../SRS/CA}

\newcommand{\progname}{Library of Simplex Method Solvers}
\newcommand{\famname}{LoSMS}

\begin{document}

\title{System Verification and Validation Plan of a Library of Simplex Method 
Solvers (\famname{}) \wss{put your library name in the title}\hz{done}}
\author{Hanane Zlitni}
\date{October 22, 2018}
	
\maketitle

\pagenumbering{roman}

\section{Revision History}

\begin{tabularx}{\textwidth}{p{3cm}p{2cm}X}
\toprule {\bf Date} & {\bf Version} & {\bf Notes}\\
\midrule
December 17, 2018 & 2.0 & Final Draft (includes making the document consistent 
with the rest of the deliverables)\\
December 16, 2018 & 1.2 & Applied Dr.~Smith’s Comments\\
December 06, 2018 & 1.1 & Applied Brooks MacLachlan’s Comments Posted on 
GitHub\\
October 22, 2018 & 1.0 & First Draft\\
\bottomrule
\end{tabularx}

~\newpage

\section{Symbols, Abbreviations and Acronyms}

\renewcommand{\arraystretch}{1.2}
\begin{tabular}{l l} 
  \toprule		
  \textbf{symbol} & \textbf{description}\\
  \midrule 
  T & Test\\
  V\&V & Verification and Validation\\
  \famname{} & Library of Simplex Method Solvers\\
  CA & Commonality Analysis\\
  SRS & Software Requirements Specification\\
  A & Assumption\\
  IM & Instance Model\\
  $s.\;t.$ & Subject to\\
  $Z$ & Optimal solution of the objective function\\
  $x_1, x_2, x_3$ & Decision variables\\
  \bottomrule
\end{tabular}\\

\newpage

\tableofcontents

\listoftables

\newpage

\pagenumbering{arabic}

This document describes the system verification and validation (V\&V) plan for
the \progname{} (\famname{}) tool. \wss{I recently modified the blank project 
template to put the (equivalent of) the famname command in a common file, which 
can be shared between all your files. You might want to do the same.}\hz{done} 
It is based on the tool's commonality analysis (CA) that can be found, along 
with the full documentation of \famname{}, at: 
\url{https://github.com/hananezlitni/HZ-CAS741-Project}.
\wss{nice to include this link}\hz{thank you} \\

The V\&V plan starts by providing general information about the tool and this 
document in Section \ref{GeneralInfo}. Then, Section \ref{Plan} provides 
additional details about the plan, which includes information about the V\&V 
team, the SRS, design and implementation verification plans and the software 
validation plan. This is followed by the system test description in Section 
\ref{SystemTestDescription}, which consists of tests for the tool's functional 
and nonfunctional requirements and traceability between test cases and 
requirements. The document is concluded by Section \ref{StaticVerTechniques} 
which describes the techniques for static verification.

\section{General Information} \label{GeneralInfo}

\subsection{Summary}

The software under test, \famname{}, is a general-purpose program family that 
facilitates obtaining the optimal solution of a linear program, using the 
simplex method, given the objective function, the objective function goal 
(maximization or minimization) and the linear constraints. Since the simplex 
algorithm is widely used in various fields, \famname{} is intended to be used 
by people from different backgrounds to help them optimize parameters of their 
choice.

\subsection{Objectives}

The objective of this verification and validation plan is to build confidence 
in the correctness of the \famname{} tool (i.e. it produces the correct output 
for the corresponding inputs), while providing satisfactory usability.

\subsection{References}

Different sections in this document refer to the tool's CA which can be found 
at: 
\url{https://github.com/hananezlitni/HZ-CAS741-Project/blob/master/docs/SRS/CA.pdf}.
\wss{You should reference your SRS here.}\hz{done}

\section{Plan} \label{Plan}
	
\subsection{Verification and Validation Team}

The verification and validation team consists of one member: Hanane Zlitni.

\subsection{SRS Verification Plan}

The CA for the \famname{} tool will be verified by getting feedback from 
Dr.~Spencer Smith \wss{\LaTeX{} has a rule that it inserts two spaces at the 
end of a sentence.  It detects a sentence as a period followed by a capital 
letter. This comes up, for instance, with Dr. Smith.  Since the period after 
Dr.\ isn't actually the end of a sentence, you need to tell \LaTeX{} to insert 
one space.  You do this either by Dr.\ Smith (if you don't mind a line-break
between Dr.\ and Smith), or Dr.~Spencer Smith (to force \LaTeX{} to not insert
a line break).}\hz{done} and my CAS 741 classmates which comes from formal 
reviews of the document. \\

The feedback from Dr.~Smith is expressed as comments written in the actual 
document. The feedback from the CAS 741 students, on the other hand, is 
expressed as issues posted in the project's GitHub repository. An Excel sheet 
``Repos.xlsx'' present in the course's GitHub repository contains the names of 
the reviewers for each documentation (Brooks MacLachlan for this document). 
After getting the comments, I apply them one by one and make changes in the 
document. \wss{You could be more specific (using Repos.xlsx) on who exactly is 
going to review your documentation.}\hz{done}

\subsection{Design Verification Plan}

\famname{}'s design documents will be verified by getting feedback from 
Dr.~Spencer Smith and my CAS 741 classmates which comes from formal reviews of 
the documents.

The feedback from Dr.~Smith is expressed as comments written in the documents, 
whereas the feedback from the CAS 741 students is expressed as issues posted in 
the project's GitHub repository. After getting the comments, I apply them one 
by one and make changes in the documents.

\subsection{Implementation Verification Plan}

The implementation of the \famname{} tool will be verified dynamically by 
executing the test cases detailed in this plan and the Unit V\&V Plan using 
testing frameworks (e.g. Python's unittest). Based on the errors I get, the 
implementation will be modified. \\

If the time permits, the implementation of the \famname{} tool will be 
statically verified by performing code review with Dr.~Spencer Smith and my CAS 
741 classmates using the GitHub issue tracker.
 
\subsection{Software Validation Plan}

Not applicable for \famname{}.

\section{System Test Description} \label{SystemTestDescription}

System testing for the \famname{} tool ensures that the correct inputs produce 
the correct outputs. The test cases in this section are derived from the 
instance models and the requirements detailed in the tool's CA, found at: 
\url{https://github.com/hananezlitni/HZ-CAS741-Project/blob/master/docs/SRS/CA.pdf}.

\subsection{Tests for Functional Requirements}

This section contains the test cases related to the tool’s functional 
requirements (e.g. handling input errors, calculating the solution and 
producing the output). They are categorized based on the linear programming 
problem goal (maximization and minimization). This section is based on the CA 
which can be found at 
\url{https://github.com/hananezlitni/HZ-CAS741-Project/blob/master/docs/SRS/CA.pdf}.

\wss{It would be nice to have a blurb here to explain why the subsections below
  cover the requirements.  References to the SRS would be good.}\hz{done}

\subsubsection{Tests for Solving Maximization Linear Programs}

The following are the test cases related to solving maximization linear 
programs. It covers both cases of having and not having an optimal solution. 
This is based on the tool's CA which can be found at 
\url{https://github.com/hananezlitni/HZ-CAS741-Project/blob/master/docs/SRS/CA.pdf}.

\hz{Note: I excluded the case where there are multiple optimal solutions 
because after many tries, I wasn't able to achieve this case in the code. I 
wanted to include a paragraph about why I excluded it, but at this point in the 
document I think we're still abstract and not thinking about the code yet.}


\wss{It would be nice to have a blurb here to explain why the subsections below
  cover the requirements.  References to the SRS would be good.}\hz{done}

\begin{enumerate}
	\item{\textbf{T\refstepcounter{testnum}\thetestnum \label{Max_Opt}: Optimal 
	Solution}}
	
	Control: Automatic 
	
	Initial State: -
	
	Input: max$\;Z\;=\;2x_1\;-\;3x_2\;+\;x_3$\\
	$s.\;t.$$\hspace*{1.3cm} x_1\;+\;x_2\;+\;x_3\; \leq \;10$\\
	$\hspace*{2cm} 4x_1\;-\;3x_2\;+\;x_3\; \leq \;3$\\
	$\hspace*{2cm} 2x_1\;+\;x_2\;-\;x_3\; \leq \;10$\\
	
	Output: $Z = 3$, $x_1=0,\;x_2=0,\;x_3=3$
	
	How test will be performed: Unit testing using Unittest
	
	Test Case Derivation: IM1 in \cite{losms-ca} \wss{good to have a field
          for test case derivation.  I might have to add that to the blank
          template.  :-)}\hz{thank you (:}

	\item{\textbf{T\refstepcounter{testnum}\thetestnum \label{Max_NoOpt}: No 
	Optimal Solution}}
	
	Control: Automatic 
	
	Initial State: -
	
	Input: max$\;Z\;=\;2x_1\;+\;x_2$\\
	$s.\;t.$$\hspace*{1.3cm} x_1\;-\;x_2\; \leq \;10$\\
	$\hspace*{2cm} 2x_1\;-\;x_2\; \leq \;40$\\
	
	Output: Exception message: ``This linear program does not have an optimal 
	solution''
	
	How test will be performed: Unit testing using Unittest
	
	Test Case Derivation: IM1 in \cite{losms-ca} and 
	\cite{simplex-special-situations}
\end{enumerate}

\subsubsection{Tests for Solving Minimization Linear Programs}

The following are the test cases related to solving minimization linear 
programs. It covers both cases of having and not having an optimal solution. 
This is based on the tool's CA which can be found at 
\url{https://github.com/hananezlitni/HZ-CAS741-Project/blob/master/docs/SRS/CA.pdf}.

\begin{enumerate}
	\item{\textbf{T\refstepcounter{testnum}\thetestnum \label{Min_Opt}: Optimal 
	Solution}}
	
	Control: Automatic 
	
	Initial State: -
	
	Input: min$\;Z\;=\;-2x_1\;+\;3x_2$\\
	$s.\;t.$$\hspace*{1.3cm} 3x_1\;+\;4x_2\; \leq \;24$\\
	$\hspace*{2cm} 7x_1\;+\;4x_2\; \leq \;16$\\
	
	Output: $Z = -4.5714$, $x_1=2.2857,\;x_2=0$
	
	How test will be performed: Unit testing using Unittest
	
	Test Case Derivation: IM2 in \cite{losms-ca}
	
	Note: Since this test case includes the comparison of floats and they're 
	usually not exactly equal, the function \textit{math.isclose()} that Python 
	3 provides will be used. By using this function, a small difference between 
	the two floats being compared will be allowed. (Source: \cite{python3-doc}).
	
	\item{\textbf{T\refstepcounter{testnum}\thetestnum \label{Min_NoOpt}: No 
	Optimal Solution}}
	
	Control: Automatic 
	
	Initial State: -
	
	Input: min$\;Z\;=\;3x_1\;+\;14x_2$\\
	$s.\;t.$$\hspace*{1.3cm} -x_1\;-\;5x_2\; \leq \;-6$\\
	$\hspace*{2cm} -x_1\;-\;4x_2\; \leq \;-5$\\
	$\hspace*{2cm} -x_1\;-\;3x_2\; \leq \;-4$\\
	$\hspace*{2cm} -x_1\;-\;2x_2\; \leq \;-5$\\
	$\hspace*{2cm} -x_1\;-\;x_2\; \leq \;-6$\\
	
	Output: Exception message: ``This linear program does not have an optimal 
	solution''
	
	How test will be performed: Unit testing using Unittest
	
	Test Case Derivation: IM2 in \cite{losms-ca}
\end{enumerate}

\subsubsection{Tests for Faulty Inputs} \label{TestInputs}

The following test cases verify that the received inputs are not empty. It is 
based on the tool's CA found at: 
\url{https://github.com/hananezlitni/HZ-CAS741-Project/blob/master/docs/SRS/CA.pdf}.\\

\begin{enumerate}
	\item{\textbf{T\refstepcounter{testnum}\thetestnum \label{NoObjcFunc}: No 
	Objective Function}}
	
	Control: Automatic 
	
	Initial State: -
	
	Input: min \\
	$s.\;t.$$\hspace*{1.3cm} 3x_1\;+\;4x_2\; \leq \;24$\\
	$\hspace*{2cm} 7x_1\;+\;4x_2\; \leq \;16$
	
	Output: Exception message: ``At least one input is missing''
	
	How test will be performed: Unit testing using Unittest
	
	Test Case Derivation: IM1 \& IM2 in \cite{losms-ca}
	
	\item{\textbf{T\refstepcounter{testnum}\thetestnum \label{NoLCs}: No 
			Linear Constraints}}
	
	Control: Automatic 
	
	Initial State: -
	
	Input: min$\;Z\;=\;-2x_1\;+\;3x_2$
	
	Output: Exception message: ``At least one input is missing''
	
	How test will be performed: Unit testing using Unittest
	
	Test Case Derivation: IM1 \& IM2 in \cite{losms-ca}
\end{enumerate}

\wss{Seems like good coverage.}

\subsection{Tests for Nonfunctional Requirements} \label{NonFunctionalTests}

\subsubsection{Usability}

\begin{enumerate}
	\item{\textbf{T\refstepcounter{testnum}\thetestnum \label{Usability}: Test 
	for the Usability of \famname{}}}
	
	Type: Usability Testing
						
	Initial State: -
						
	Input/Condition: -
						
	Output/Result: -
						
	How test will be performed: For testing purposes, I will be implementing a 
	command line handling tool that will use \famname{}. For the usability 
	testing, I will be asking participants to try the tool \wss{you should 
	clarify that you are going to make a tool using your library}\hz{done} then 
	answer the usability survey questions (see Section \ref{UsabilityTesting}). 
	The goal is to ensure that the library provided the services the users 
	requested and that they are satisfied with the results obtained.
\end{enumerate}

\subsubsection{Portability}

\begin{enumerate}
	\item{\textbf{T\refstepcounter{testnum}\thetestnum \label{Portability}: 
	Test for the Portability of \famname{}}}
	
	Type: Static
	
	Initial State: -
	
	Input/Condition: -
	
	Output/Result: -
	
	How test will be performed: Running \famname{} on Mac, Windows and Linux 
	operating systems
\end{enumerate}

\subsubsection{Accuracy}

\begin{enumerate}
	\item{\textbf{T\refstepcounter{testnum}\thetestnum \label{Accuracy}: Test 
	for the Accuracy of the Outputs}}

	Type: Dynamic
	
	Initial State: -
	
	Input/Condition: -
	
	Output/Result: -
	
	How test will be performed: I plan to report the relative error of the 
	expected output for each test case detailed in this document
        \wss{great!}
\end{enumerate}

\subsubsection{Correctness}

\begin{enumerate}
	\item{\textbf{T\refstepcounter{testnum}\thetestnum \label{Correctness}: 
	Test for the Correctness of \famname{}}}
	
	Type: Parallel Testing
	
	Initial State: -
	
	Input/Condition: -
	
	Output/Result: -
	
	How test will be performed: I plan to make a comparison between \famname{} 
	and MatLab to evaluate the correctness of \famname{}. The problem to be 
	tested is:
	
	min$\;Z\;=\;-x_1\;-\;2x_2$\newline
	$s.\;t.$$\hspace*{0.5cm} 2x_1\;+\;x_2\; = \;10$\newline
	$\hspace*{1.2cm} x_1\;+\;x_2\; = \;6$\newline
	$\hspace*{1.2cm} -x_1\;+\;x_2\; = \;2$\newline
	$\hspace*{1.2cm} -2x_1\;+\;x_2\; = \;1$\newline
\end{enumerate}

\subsubsection{Performance}

\begin{enumerate}
	\item{\textbf{T\refstepcounter{testnum}\thetestnum \label{Performance}: 
			Test for the Performance of \famname{}}}
	
	Type: Parallel Testing
	
	Initial State: -
	
	Input/Condition: -
	
	Output/Result: -
	
	How test will be performed: I plan to make a comparison between \famname{} 
	and MatLab to evaluate the performance of \famname{} by finding the time 
	(in seconds) it took to solve a linear program. The problem to be tested is:

	min$\;Z\;=\;-x_1\;-\;2x_2$\newline
	$s.\;t.$$\hspace*{0.5cm} 2x_1\;+\;x_2\; = \;10$\newline
	$\hspace*{1.2cm} x_1\;+\;x_2\; = \;6$\newline
	$\hspace*{1.2cm} -x_1\;+\;x_2\; = \;2$\newline
	$\hspace*{1.2cm} -2x_1\;+\;x_2\; = \;1$\newline
\end{enumerate}

\subsubsection{Stability}

\begin{enumerate}
	\item{\textbf{T\refstepcounter{testnum}\thetestnum \label{Stability}: Test 
	for the Stability of \famname{} Under Heavy Load}}
	
	Type: Stress/Load Testing
	
	Initial State: -
	
	Input/Condition: -
	
	Output/Result: -
	
	How test will be performed: I plan to use a third-party tool that automates 
	loading \famname{} with inputs so I can reach the greatest possible number 
	of inputs (i.e.~hundreds). Specifically, I plan to increase the number 
	of parameters in the objective function and the number of linear 
	constraints to check the stability of the library. \wss{You need to be more 
	specific.}\hz{done. I tried to be as specific as possible and answer the 
	questions given by Brooks in the issue tracker.}

\end{enumerate}

\subsection{Traceability Between Test Cases and Requirements}

The following table describes the mapping between the test cases and 
requirements.

\begin{table} [h!]
	\centering
	\begin{tabular}{|c|c|}
		\hline	
		\textbf{Test Case Number} & \textbf{Requirements}\\
		\hline 
		T\ref{Max_Opt}& R\ref{R_Inputs}, R\ref{R_StandardForm}, 
		R\ref{R_CanonicalForm}, R\ref{R_Calculate}, R\ref{R_Output}\\ \hline
		T\ref{Max_NoOpt}& R\ref{R_Inputs}, R\ref{R_StandardForm}, 
		R\ref{R_CanonicalForm}, R\ref{R_Calculate}, R\ref{R_Output}\\ \hline
		T\ref{Min_Opt}& R\ref{R_Inputs}, R\ref{R_StandardForm}, 
		R\ref{R_CanonicalForm}, R\ref{R_Calculate}, R\ref{R_Output}\\ \hline
		T\ref{Min_NoOpt}& R\ref{R_Inputs}, R\ref{R_StandardForm}, 
		R\ref{R_CanonicalForm}, R\ref{R_Calculate}, R\ref{R_Output}\\ \hline
		T\ref{NoObjcFunc}& R\ref{R_Inputs}, R\ref{R_HandleInputErrors}, 
		R\ref{R_DisplayErrorMsg}\\ \hline
		T\ref{NoLCs}& R\ref{R_Inputs}, R\ref{R_HandleInputErrors}, 
		R\ref{R_DisplayErrorMsg}\\ \hline
		T\ref{Usability}& NFR\ref{NFR_usability}\\ \hline
		T\ref{Portability}& NFR\ref{NFR_portability}\\ \hline
		T\ref{Accuracy}& NFR\ref{NFR_accuracy}\\ \hline
		T\ref{Correctness}& NFR\ref{NFR_correctness}\\ \hline
		T\ref{Performance}& NFR\ref{NFR_performance}\\ \hline
		T\ref{Stability}& NFR\ref{NFR_stability}\\ \hline
	\end{tabular}
	\caption{Traceability Between Test Cases and Requirements}
	\label{Table:TraceabilityTandR} 
\end{table}

\FloatBarrier

\section{Static Verification Techniques} \label{StaticVerTechniques}

If the time permits, static verification of the \famname{} library 
implementation will performed using code review with Dr.~Spencer Smith and my 
CAS 741 classmates (the comments will be in the form of GitHub issues).

\newpage
				
\bibliographystyle {plainnat}
\bibliography {../../../refs/References}

\newpage

\section{Appendix}

This section provides additional content related to this system V\&V plan.

\subsection{Symbolic Parameters}

There are no symbolic parameters used in this document.

\subsection{Usability Survey Questions} \label{UsabilityTesting}

\begin{enumerate}
	\item Did \famname{} successfully provide all the services you requested?
	
	(	Yes	/	No	)
	
	Why have you chosen the above response?
	
	\item How confident are you that \famname{} provided you with the correct 
	results?
	
	(	1	/	2	/	3	/	4	/	5	) ; \textit{(1) being not confident 
	at all and (5) being very confident}

	Why have you given the above score?

	\item How satisfied are you with the library's response time?  
	
	(	1	/	2	/	3	/	4	/	5	) ; \textit{(1) being not satisfied 
	at all and (5) being very satisfied}

	Why have you given the above score?

	\item How likely are you to recommend \famname{} to a friend?
	
	(	1	/	2	/	3	/	4	/	5	) ; \textit{(1) being very unlikely 
	and (5) being very likely}

	Why have you given the above score?

	\item Rate your overall satisfaction with \famname{} out of 10 
	
	(	1	/	2	/	3	/	4	/	5	/	6	/	7	/	8	/	9	/	
	10	)
	
	Why have you given the above score?
\end {enumerate}

\wss{Good survey questions.}\hz{thank you}

\end{document}