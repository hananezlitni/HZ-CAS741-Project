\documentclass[12pt, titlepage]{article}

\usepackage{booktabs}
\usepackage{tabularx}
\usepackage{hyperref}
\hypersetup{
    colorlinks,
    citecolor=blue,
    filecolor=magenta,
    linkcolor=black,
    urlcolor=cyan
}
\usepackage[round]{natbib}



\input{../../Comments}

\newcommand{\famname}{LoSMS} % PUT YOUR PROGRAM NAME %HERE

\begin{document}

\title{A Library of Simplex Method Solvers: System Verification and Validation 
Plan} 
\author{Hanane Zlitni}
\date{October 22, 2018}
	
\maketitle

\pagenumbering{roman}

\section{Revision History}

\begin{tabularx}{\textwidth}{p{3cm}p{2cm}X}
\toprule {\bf Date} & {\bf Version} & {\bf Notes}\\
\midrule
October 18, 2018 & 1.0 & Presentation Draft\\
\bottomrule
\end{tabularx}

~\newpage

\section{Symbols, Abbreviations and Acronyms}

\renewcommand{\arraystretch}{1.2}
\begin{tabular}{l l} 
  \toprule		
  \textbf{symbol} & \textbf{description}\\
  \midrule 
  T & Test\\
  V\&V & Verification and Validation\\
  \famname{} & A Library of Simplex Method Solvers\\
  CA & Commonality Analysis\\
  \bottomrule
\end{tabular}\\

\wss{symbols, abbreviations or acronyms -- you can simply reference the SRS
  tables, if appropriate}

\newpage

\tableofcontents

\listoftables

\listoffigures

\newpage

\pagenumbering{arabic}

This document describes the system verification and validation plan (V\&V) for 
the \famname{} (Library of Simplex Method Solvers) tool. It is based on the 
tool's commonality analysis (CA) that can be found, along with the full 
documentation of \famname{}, in the following link:  
\url{https://github.com/hananezlitni/HZ-CAS741-Project}.

The V\&V plan starts by providing general information about the tool and this 
document in Section \ref{GeneralInfo}. Then, Section \ref{Plan} provides 
additional details about the plan, which includes information about the V\&V 
team, the SRS, design and implementation verification plans and the software 
validation plan. This is followed by the system test description in Section 
\ref{SystemTestDescription}, which consists of tests for the tool's functional 
and nonfunctional requirements and traceability between test cases and 
requirements. Finally, the document is concluded by Section 
\ref{StaticVerTechniques} which describes the techniques for static 
verification.

\section{General Information} \label{GeneralInfo}

\subsection{Summary}

The software under test, \famname{}, is a general-purpose program family that 
facilitates obtaining the optimal solution of a linear program, using the 
simplex method, given the objective function, the objective function goal 
(maximization or minimization) and the linear constraints. Since the simplex 
algorithm is widely used in various fields, \famname{} is intended to be used 
by people from different backgrounds to help them optimize parameters of their 
choice. \hz{get back to this}

\subsection{Objectives}

\wss{State what is intended to be accomplished.  The objective will be around
  the qualities that are most important for your project.  You might have
  something like: ``build confidence in the software correctness,''
  ``demonstrate adequate usability.'' etc.  You won't list all of the qualities,
  just those that are most important.}

\subsection{References}

\begin{itemize}
	\item CA
	\item "Special situations in Simplex method" paper
\end{itemize}

\section{Plan} \label{Plan}
	
\subsection{Verification and Validation Team}

The verification and validation team consists of one member: Hanane Zlitni.

\subsection{SRS Verification Plan}

The CA for the \famname{} tool will be verified by getting feedback from Dr. 
Spencer Smith and my CAS 741 classmates.

\subsection{Design Verification Plan}

\famname{}'s design documents will be verified by getting feedback from 
Dr. Spencer Smith and my CAS 741 classmates.

\subsection{Implementation Verification Plan}

The implementation of the \famname{} tool will be verified statically by 
performing code review with Dr. Spencer Smith and my CAS 741 classmates and 
dynamically by executing the test cases detailed in this plan and the unit V\&V 
plan using testing frameworks (e.g. JUnit/PyUnit).
 
\subsection{Software Validation Plan}

Not applicable for \famname{}.

\section{System Test Description} \label{SystemTestDescription}

System testing for the \famname{} tool ensures that the correct inputs produce 
the correct outputs. The test cases in this section are derived from the 
instance models and the requirements detailed in the tool's CA.
	
\subsection{Tests for Functional Requirements}

\subsubsection{Tests for Solving Maximization Linear Programs}

\begin{enumerate}
	\item{\textbf{T1: Unique Optimal Solution}}
	
	Control: Automatic 
	
	Initial State: -
	
	Input: $max\;Z\;=\;2x_1\;-\;3x_2\;+\;x_3$\\
	$s.\;t.$$\hspace*{1.3cm} x_1\;+\;x_2\;+\;x_3\; \leq \;10$\\
	$\hspace*{2cm} 4x_1\;-\;3x_2\;+\;x_3\; \leq \;3$\\
	$\hspace*{2cm} 2x_1\;+\;x_2\;-\;x_3\; \leq \;10$\\
	$\hspace*{2cm} x_1\;,\;x_2\;,\;x_3\; \geq \;0$
	
	Output: $Z = 3$, occurring when $x_1=0,\;x_2=0,\;x_3=3$
	
	How test will be performed: Unit testing using JUnit/PyUnit
	
	Test Case Derivation: \hz{reference IM1 in CA}

	\item{\textbf{T2: Multiple Optimal Solutions}}
	
	Control: Automatic 
	
	Initial State: -
	
	Input: $max\;Z\;=\;3x_1\;+\;2x_2$\\
	$s.\;t.$$\hspace*{1.3cm} 3x_1\;+\;2x_2\; \leq \;180$\\
	$\hspace*{2cm} x_1\; \leq \;40$\\
	$\hspace*{2cm} x_2\; \leq \;60$\\
	$\hspace*{2cm} x_1\;,\;x_2\; \geq \;0$
	
	Output: $Z = 180$, occurring when $x_1=40,\;x_2=30$ \& $x_1=20,\;x_2=60$
	
	How test will be performed: Unit testing using JUnit/PyUnit
	
	Test Case Derivation: \hz{reference IM1 in CA}

	\item{\textbf{T3: No Optimal Solution}}
	
	Control: Automatic 
	
	Initial State: -
	
	Input: $max\;Z\;=\;2x_1\;+\;x_2$\\
	$s.\;t.$$\hspace*{1.3cm} x_1\;-\;x_2\; \leq \;10$\\
	$\hspace*{2cm} 2x_1\;-\;x_2\; \leq \;40$\\
	$\hspace*{2cm} x_1\;,\;x_2\; \geq \;0$
	
	Output: "This linear program does not have an optimal solution"
	
	How test will be performed: Unit testing using JUnit/PyUnit
	
	Test Case Derivation: \hz{reference IM1 in CA}
\end{enumerate}

\subsubsection{Tests for Solving Minimization Linear Programs}

\begin{enumerate}
	\item{\textbf{T4: Unique Optimal Solution}}
	
	Control: Automatic 
	
	Initial State: -
	
	Input:
	
	Output: 
	
	How test will be performed: Unit testing using JUnit/PyUnit
	
	Test Case Derivation: 
	
	\item{\textbf{T5: Multiple Optimal Solutions}}
	
	Control: Automatic 
	
	Initial State: -
	
	Input: 
	
	How test will be performed: 
	
	Test Case Derivation: 
	
	\item{\textbf{T6: No Optimal Solution}}
	
	Control: Automatic 
	
	Initial State: -
	
	Input:
	
	Output: 
	
	How test will be performed: Unit testing using JUnit/PyUnit
	
	Test Case Derivation: 
\end{enumerate}

\subsubsection{Tests for Faulty Inputs}

\begin{enumerate}
	\item{\textbf{T7: No Objective Function}}
	
	Control: Automatic
	
	Initial State: -
	
	Input: $min $\\
	$s.\;t.$$\hspace*{1.3cm} x_1\;+\;x_2\;+\;x_3\; \leq \;10$\\
	$\hspace*{2cm} 4x_1\;-\;3x_2\;+\;x_3\; \leq \;3$\\
	$\hspace*{2cm} 2x_1\;+\;x_2\;-\;x_3\; \leq \;10$\\
	$\hspace*{2cm} x_1\;,\;x_2\;,\;x_3\; \geq \;0$
	
	Output: "Error: No objective function"
	
	How test will be performed: Unit testing using JUnit/PyUnit
	
	Test Case Derivation: \hz{reference IM1 \& IM2 in CA}
	
	\item{\textbf{T8: No Linear Constraints}}
	
	Control: Automatic
	
	Initial State: -
	
	Input: $max\;Z\;=\;2x_1\;-\;3x_2\;+\;x_3$
	
	Output: "Error: No linear constraints"
	
	How test will be performed: Unit testing using JUnit/PyUnit
	
	Test Case Derivation: \hz{reference IM1 \& IM2 in CA}
	
	\item{\textbf{T9: No Objective Function Goal}}
	
	Control: Automatic
	
	Initial State: -
	
	Input: $\;Z\;=\;2x_1\;-\;3x_2\;+\;x_3$\\
	$s.\;t.$$\hspace*{1.3cm} x_1\;+\;x_2\;+\;x_3\; \leq \;10$\\
	$\hspace*{2cm} 4x_1\;-\;3x_2\;+\;x_3\; \leq \;3$\\
	$\hspace*{2cm} 2x_1\;+\;x_2\;-\;x_3\; \leq \;10$\\
	$\hspace*{2cm} x_1\;,\;x_2\;,\;x_3\; \geq \;0$
	
	Output: "Error: No objective function goal"
	
	How test will be performed: Unit testing using JUnit/PyUnit
	
	Test Case Derivation: \hz{reference IM1 \& IM2 in CA}
	
	\item{\textbf{T10: No Non-negativity Constraints/Negative Decision 
	Variables}}
	
	Control: Automatic
	
	Initial State: -
	
	Input: $max\;Z\;=\;2x_1\;-\;3x_2$\\
	$s.\;t.$$\hspace*{1.3cm} x_1\;+\;x_2\; \leq \;5$\\
	$\hspace*{2cm} 4x_1\;-\;3x_2\; \leq \;4$
	
	Output: "Error: The decision variables must be positive"
	
	How test will be performed: Unit testing using JUnit/PyUnit
	
	Test Case Derivation: \hz{reference T1 in CA}
	
	\item{\textbf{T11: Greater Than or Equal to Inequalities}}
	
	Control: Automatic
	
	Initial State: -
	
	Input: $max\;Z\;=\;2x_1\;-\;3x_2$\\
	$s.\;t.$$\hspace*{1.3cm} x_1\;+\;x_2\; \leq \;5$\\
	$\hspace*{2cm} 4x_1\;-\;3x_2\; \geq \;4$\\
	$\hspace*{2cm} x_1\;,\;x_2\; \geq \;0$
	
	Output: "Error: The inequalities of the main constraints must be of type 
	less than or equal to"
	
	How test will be performed: Unit testing using JUnit/PyUnit
	
	Test Case Derivation: \hz{reference A2 in CA}
\end{enumerate}

\subsection{Tests for Nonfunctional Requirements}

\subsubsection{Usability}

\begin{enumerate}
	\item{\textbf{T12: Test for the Usability of \famname{}}}
	
	Type: Usability Testing
						
	Initial State: -
						
	Input/Condition: -
						
	Output/Result: -
						
	How test will be performed: Asking participants to try the tool then answer 
	the usability survey questions (see Appendix \ref{UsabilityTesting})
\end{enumerate}

\subsubsection{Portability}

\begin{enumerate}
	\item{\textbf{T13: Test for the Portability of \famname{}}}
	
	Type: Static \hz{??}
	
	Initial State: -
	
	Input/Condition: -
	
	Output/Result: -
	
	How test will be performed: Running \famname{} on different operating 
	systems
\end{enumerate}

\subsubsection{Robustness}

\begin{enumerate}
	\item{\textbf{T14: Test for the Robustness of \famname{}}}
	
	Type: Dynamic \hz{??}
	
	Initial State: -
	
	Input/Condition: -
	
	Output/Result: -
	
	How test will be performed: \hz{mutation testing?}
\end{enumerate}

\subsection{Traceability Between Test Cases and Requirements}

\wss{Provide a table that shows which test cases are supporting which
  requirements.}

\section{Static Verification Techniques} \label{StaticVerTechniques}

\wss{In this section give the details of any plans for static verification of
  the implementation.  Potential techniques include code walkthroughs, code
  inspection, static analyzers, etc.}

\newpage
				
\bibliographystyle{plainnat}

\bibliography{SRS}

\newpage

\section{Appendix}

This section provides additional content related to this system V\&V plan.

\subsection{Symbolic Parameters}

There are no symbolic parameters used in this document.

\subsection{Usability Survey Questions} \label{UsabilityTesting}

\begin{enumerate}
	\item Overall, how easy to use do you find \famname{}? \hz{range 1-5}
	\item Overall, how confident are you that you completed the task 
	successfully? \hz{range 1-5}
	\item Rate your satisfaction with \famname{} out of 10
	
	\hz{add more questions}
\end {enumerate}

\end{document}