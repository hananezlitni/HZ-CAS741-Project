\documentclass[12pt, titlepage]{article}

\usepackage{amsmath, mathtools}

\usepackage[round]{natbib}
\usepackage{amsfonts}
\usepackage{amssymb}
\usepackage{graphicx}
\usepackage{colortbl}
\usepackage{xcolor}
\usepackage{xr}
\usepackage{hyperref}
\usepackage{longtable}
\usepackage{xfrac}
\usepackage{tabularx}
\usepackage{float}
\usepackage{siunitx}
\usepackage{booktabs}
\usepackage{multirow}
\usepackage[section]{placeins}
\usepackage{caption}
\usepackage{fullpage}

\hypersetup{
bookmarks=true,     % show bookmarks bar?
colorlinks=true,       % false: boxed links; true: colored links
linkcolor=black,          % color of internal links (change box color with 
%linkbordercolor)
citecolor=blue,      % color of links to bibliography
filecolor=magenta,  % color of file links
urlcolor=cyan          % color of external links
}

\usepackage{array}

\externaldocument{../../SRS/CA}

\input{../../Comments}

\newcommand{\progname}{Library of Simplex Method Solvers}
\newcommand{\famname}{LoSMS}
\definecolor{pyComment}{RGB}{179, 179, 204}

\begin{document}

\title{Module Interface Specification for a \progname{} (\famname{})}

\author{Hanane Zlitni}

\date{November 24, 2018}

\maketitle

\pagenumbering{roman}

\section{Revision History}

\begin{tabularx}{\textwidth}{p{3cm}p{2cm}X}
\toprule {\bf Date} & {\bf Version} & {\bf Notes}\\
\midrule
December 17, 2018 & 2.0 & Final Draft (includes making the document consistent 
with the rest of the deliverables)\\
December 16, 2018 & 1.2 & Applied Dr.~Smith’s Comments\\
December 06, 2018 & 1.2 & Applied Olu Owojaiye's Comments Posted on GitHub\\
December 03, 2018 & 1.1 & Corrected solveLP() and pivot() pseudo-code in 8.4.4 
and 8.4.5\\
November 24, 2018 & 1.0 & First draft\\
\bottomrule
\end{tabularx}

~\newpage

\section{Symbols, Abbreviations and Acronyms}

See SRS Documentation at 
\url{https://github.com/hananezlitni/HZ-CAS741-Project/blob/master/docs/SRS/CA.pdf}.
 \\

The following are additional symbols, abbreviations or acronyms used in this 
document: \\

\renewcommand{\arraystretch}{1.2}
\begin{tabular}{l l} 
	\toprule		
	\textbf{symbol} & \textbf{description}\\
	\midrule 
	$Z$ & Optimal solution(s) of the objective function\\
	$Z'$ & The negation of the objective function\\
	$n$ & A number in [0, $\mathbb{N}$) representing the rows in the simplex 
	tableau \\
	$m$ & A number in [0, $\mathbb{N}$) representing the columns in the 
	simplex tableau \\
	$x$ & A number in $\mathbb{N}$ representing the size of the list of 
	optimal solutions \\
	\bottomrule
\end{tabular}\\

\newpage

\tableofcontents

\newpage

\pagenumbering{arabic}

\section{Introduction}

The following document details the Module Interface Specifications for the 
\progname{} (\famname{}) tool.

Complementary documents include the System Requirement Specifications and 
Module Guide.  The full documentation and implementation can be found at  
\url{https://github.com/hananezlitni/HZ-CAS741-Project}.

\section{Notation}

The structure of the MIS for modules comes from \citet{HoffmanAndStrooper1995},
with the addition that template modules have been adapted from
\cite{GhezziEtAl2003}.  The mathematical notation comes from Chapter 3 of
\citet{HoffmanAndStrooper1995}.  For instance, the symbol := is used for a
multiple assignment statement and conditional rules follow the form $(c_1
\Rightarrow r_1 | c_2 \Rightarrow r_2 | ... | c_n \Rightarrow r_n )$.

The following table summarizes the primitive data types used by \famname{}. 

\begin{center}
\renewcommand{\arraystretch}{1.2}
\noindent 
\begin{tabular}{l l p{7.5cm}} 
\toprule 
\textbf{Data Type} & \textbf{Notation} & \textbf{Description}\\ 
\midrule
character & char & a single symbol or digit\\
integer & $\mathbb{Z}$ & a number without a fractional component in (-$\infty$, $\infty$) \\
natural number & $\mathbb{N}$ & a number without a fractional component in [1, $\infty$) \\
real & $\mathbb{R}$ & any number in (-$\infty$, $\infty$)\\
\bottomrule
\end{tabular} 
\end{center}

\noindent
The specification of \famname{} uses some derived data types: sequences, 
strings, and tuples. Sequences are lists filled with elements of the same data 
type. Strings are sequences of characters. Tuples contain a list of values, 
potentially of different types. In addition, \famname{} uses functions, which
are defined by the data types of their inputs and outputs. Local functions are
described by giving their type signature followed by their specification.

\section{Module Decomposition}

The following table is taken directly from the Module Guide document for this project.

\begin{table}[h!]
	\centering
	\begin{tabular}{p{0.3\textwidth} p{0.6\textwidth}}
		\toprule
		\textbf{Level 1} & \textbf{Level 2}\\
		\midrule
		
		{Hardware-Hiding Module} & ~ \\
		\midrule
		
		{Behaviour-Hiding Module} & ~ \\
		\midrule
		
		\multirow{3}{0.3\textwidth}{Software Decision Module}
		& Simplex Method Solver\\
		& Exceptions\\
		\bottomrule
	\end{tabular}
	\caption{Module Hierarchy}
	\label{TblMH}
\end{table}

\newpage
~\newpage

\section{MIS of the Simplex Method Solver Module} \label{M_SimplexSolver} 

\subsection{Module}

SimplexSolverADT

\subsection{Uses}

Exceptions

\subsection{Syntax}

\subsubsection{Exported Constants}

None

\subsubsection{Exported Access Programs}

\begin{center}
	\begin{tabular}{p{4cm} p{4cm} p{3cm} p{5cm}}
		\hline
		\textbf{Name} & \textbf{In} & \textbf{Out} & \textbf{Exceptions} \\
		\hline
		solveLP & $\mathbb{R} ^{n*m}$, $\mathbb{R} ^{n}$, $\mathbb{R} ^{m}$, 
		$\mathbb{R} ^{m}$, gEnum & - & NO\_OPTIMAL\_SOLUTION \\
		formSimplexTableau & - & - & - \\
		toCanonical & - & - & - \\
		isOptimumFound & - & boolean & - \\
		getOptimum & - & $\mathbb{R} ^{i}$ & - \\
		\hline
	\end{tabular}
\end{center}

\hz{There is no need to define a variable for the linear constraints type 
because whether it was an equality or inequality, slack/artificial variables 
would be added}

\wss{This seems like important information to put in comments.  Comments are not
  displayed with the final documentation, so important information needs to be
  included with the main text.}\hz{done.}

\wss{For these types you are defining, \textit{lctEnum} and \textit{gEnum},
  please do not use integers that map to something meaningful.  Mathematically
  you can define a set for elements of this type.  A language like Python gives
  you enumerated types, so you can implement your new types easily.  Hoffmann
  and Strooper show simple examples of defining new types.  You will have
  something like \textit{gEnum} $= \{ \text{Min}, \text{Max} \}$.  You can
  export these types, so that other modules can use them.  If there are many new
types like this, you can create a new modules whose purpose is to export the
``global'' types that you will be using.}

\wss{My comments about these explanation comments and types applies throughout
  your document.}

\subsection{Semantics}

\subsubsection{State Variables}

\begin{itemize}
	\item objcFunc:$\mathbb{R} ^m$
	\item LCs:$\mathbb{R} ^{n*m}$
	\item constants:$\mathbb{R} ^{m}$
	\item goal:gEnum ; where gEnum = $\{MAX,MIN\}$
	\item sTableau:TableauT
	\item wasMin:boolean
\end{itemize}

\subsubsection{Explanation}

\textit{gEnum} is an enumerated type representing the goal of the linear 
program: MAX for maximization and MIN for minimization. \\

The variable \textit{wasMin} is a way to tell whether the original linear 
program was a minimization problem or not. If it's a minimization problem, the 
\textit{goal} state variable will be changed to max and \textit{wasMin} will be 
set to true. Then, the variable will be checked before outputting the solution- 
if it's True, the optimal solution $Z$ will be multiplied by -1. Otherwise, it 
will remain as it is.

\subsubsection{Environment Variables}

None

\subsubsection{Assumptions}

None

\subsubsection{Access Routine Semantics}

\noindent 
solveLP(\textit{LCs}, \textit{constants}, \textit{objcFunc}, \textit{LCsType}, 
\textit{goal}):
\begin{itemize}
	\item transition: 
	\begin{enumerate}
		\item ($\neg$(self.goal = MAX) $\Rightarrow$ self.wasMin = True, 
		self.objcFunc = self.objcFunc * -1, self.goal = MAX)
		\item self.formSimplexTableau()
		\item self.updateEnteringDepartingVars(self.sTableau)
		\item self.findPivot()
		\item (pivot $<$ 0 $\Rightarrow$ NO\_OPTIMAL\_SOLUTION)
		\item self.pivot(pivotIndex)
		\item Repeat 4, 5 and 6 until there are no negative values in the last 
		row (excluding the last column)
		\item self.getOptimum()
	\end{enumerate}
	
	\item output: -
	\item exception: \textit{exc} := 
	
	(self.getOptimum() = 0 $\vee$ self.getOptimum() = ``'' $\Rightarrow$ 
	NO\_OPTIMAL\_SOLUTION)
\end{itemize}

\noindent
formSimplexTableau():
\begin{itemize}
	\item transition:
	\begin{enumerate}
		\item Initialize \textit{sTableau}
		\item Update its state by appending \textit{LCs} to it
		\item Call toCanonical() to convert the tableau to the canonical form 
		by adding the slack/artificial variables
		\item Update the state of \textit{sTableau} by appending 
		\textit{objcFunc} to its bottom row
		\item Update the state of \textit{sTableau} by appending 
		\textit{constants} to its last column
	\end{enumerate}
	\item output: -
	\item exception: -
\end{itemize}

\noindent
toCanonical():
\begin{itemize}
	\item transition:
	\begin{enumerate}
		\item Initialize \textit{slackVariables} matrix
		\item Initialize \textit{row} list
		\item Compare between two counters \textit{i} and \textit{j}
		\item (\textit{i} = \textit{j} $\Rightarrow$ append 1 to \textit{row} 
		$|$ \textit{i} $\neq$ \textit{j} $\Rightarrow$ append 0 to \textit{row})
		\item Repeat 3 and 4 for the length of \textit{sTableau}
		\item Append \textit{row} to \textit{slackVariables}
		\item Update the state of \textit{sTableau} by appending 
		\textit{slackVariables} to it
	\end{enumerate}
	\item output: -
	\item exception: -
\end{itemize}

\noindent
getOptimum():
\begin{itemize}
	\item transition: 
	\begin{enumerate}
		\item Initialize the tuple \textit{optimum}
		\item Get the optimal solution and the points they occur from 
		\textit{sTableau} and set \textit{optimum} to the values
		\item (wasMin = True $\Rightarrow$ optimum[`z'] = optimum[`z'] * -1)	
	\end{enumerate}
	\item output: \textit{out} := optimum
	\item exception: -
\end{itemize}

\noindent
isOptimumFound()
\begin{itemize}
	\item transition:
	\begin{enumerate}
		\item Initialize the variable \textit{optimumFound} and set it to True
		\item Look for negative values in the last row of  \textit{sTableau} 
		excluding the last column and set \textit{optimumFound} to False if 
		found
	\end{enumerate}
	\item output: \textit{out} := optimumFound 
	\item exception: -
\end{itemize}

\subsubsection{Local Functions}

\noindent 
getEnteringVariable():

\textcolor{pyComment}{\#\textit{find the most negative value in the last row of 
sTableau}} \\

\hspace*{0.3cm} start \\
\hspace*{2cm} initialize \textit{lastRow} \\
\hspace*{2cm} \textit{lastRow} = self.sTableau[len(sTableau) - 1] \\
\hspace*{2cm} initialize \textit{iMostNegative} \\
\hspace*{2cm} \textit{iMostNegative} = 0 \\
\hspace*{2cm} initialize \textit{mostNegative} \\
\hspace*{2cm} \textit{mostNegative} = \textit{lastRow[iMostNegative]} \\
\hspace*{2cm} for each \textit{entry} in \textit{lastRow} \\
\hspace*{3cm} if \textit{entry} $<$ \textit{mostNegative}\\
\hspace*{4cm} if \textit{mostNegative} = \textit{entry}\\
\hspace*{3cm} update \textit{iMostNegative}\\
\hspace*{2cm} return \textit{iMostNegative} \\
\hspace*{0.9cm} end \\

\noindent 
findPivot():

\hspace*{0.3cm} start \\
\hspace*{2cm} return \textit{self.getEnteringVariable()}, 
\textit{self.getDepartingVariable()} \\
\hspace*{0.9cm} end \\

\noindent 
pivot(\textit{iPivot}):

\hspace*{0.3cm} start \\
\hspace*{2cm} set \textit{i, j} to \textit{iPivot} \\
\hspace*{2cm} set \textit{pivot} to \textit{sTableau[i][j]} \\
\hspace*{2cm} for each row in pivot column \\
\hspace*{3cm} perform a row operation to make entry 0 \\
\hspace*{0.9cm} end \\

\wss{You shouldn't be defining types here.  Seeing that these types are coming
  up again, I suggest you add a module to your design that exports types.}

\wss{Why do you have two modules with the same state variables (goal etc.).
  This makes it seem like the design is not completely thought out.  Why can't
  your Tableau module use  your input module to get the values it needs?  Or
  maybe you don't need an input module.  As we discussed in class, most of your
  design could be encapsulated in the tableau module.}

\wss{Add a brief statement on what each state variables is.  I remember what $Z$
  is, but I forget what $K$ means.}

\wss{Using the types $\mathbb{R}^x$ is confusing, since $x$ usually represents a
  real value.}

\wss{Would it be easier if you added a solver method to your tableau module?}

\hz{I have updated the MIS to reflect the changes in the design.}

~\newpage

\section{MIS of the Exceptions Module} \label{M_Exceptions} 

\subsection{Module}

Exceptions

\subsection{Uses}

None

\subsection{Syntax}

\subsubsection{Exported Constants}

None

\subsubsection{Exported Access Programs}

\begin{center}
	\begin{tabular}{p{6cm} p{4cm} p{3cm} p{3cm}}
		\hline
		\textbf{Name} & \textbf{In} & \textbf{Out} & \textbf{Exceptions} \\
		\hline
		MISSING\_INPUT & Error & - & - \\
		NO\_OPTIMAL\_SOLUTION & Error & - & - \\
		\hline
	\end{tabular}
\end{center}

\subsection{Semantics}

\subsubsection{State Variables}

None

\subsubsection{Environment Variables}

None

\subsubsection{Assumptions}

None

\subsubsection{Access Routine Semantics}

\noindent 
MISSING\_INPUT():
\begin{itemize}
	\item transition:
	\item output: \textit{out} := ``At least one input is missing.''
	\item exception: -
\end{itemize}

\noindent 
NO\_OPTIMAL\_SOLUTION():
\begin{itemize}
	\item transition:
	\item output: \textit{out} := ``This linear program does not have an 
	optimal solution.''
	\item exception: -
\end{itemize}

\subsubsection{Local Functions}

None

\newpage

\bibliographystyle {plainnat}
\bibliography {../../../refs/References}

\newpage

\section{Appendix} \label{Appendix}

There are no additional information to provide.

\end{document}